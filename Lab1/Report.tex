\documentclass{article}
\usepackage[a4paper]{geometry}
\usepackage[version=3]{mhchem} % Package for chemical equation typesetting
\usepackage{graphicx} % Required for the inclusion of images
\graphicspath{ {Images/} }
\geometry{tmargin=2.5cm,bmargin=2.5cm,lmargin=2.5cm,rmargin=2.5cm}
\usepackage{subfig}
\usepackage{float}
\usepackage{natbib} % Required to change bibliography style to APA
\usepackage{amsmath} % Required for some math elements 

%\setlength\parindent{0pt} % Removes all indentation from paragraphs

\renewcommand{\labelenumi}{\alph{enumi}.} % Make numbering in the enumerate environment by letter rather than number (e.g. section 6)

%\usepackage{times} % Uncomment to use the Times New Roman font

%----------------------------------------------------------------------------------------
%	DOCUMENT INFORMATION
%----------------------------------------------------------------------------------------

\title{Scene Segmentation and Interpretation\\Lab \#1 Region Growing} % Title

\author{Fl\'{a}via \textsc{Dias Casagrande} and Dragutin \textsc{Vujovic}} % Author name

\date{\today} % Date for the report

\begin{document}

\maketitle % Insert the title, author and date


%----------------------------------------------------------------------------------------
%	SECTION 1
%----------------------------------------------------------------------------------------

\section{Objective}

In this coursework the main goal is to develop a Region Growing algorithm for segmenting an image. The implemented algorithm should work for grey level images and colour images. The results have to be compared with the result of the Fuzzy C-Means (FCM) clustering algorithm (mostly provided in Matlab) on the same images.
 
%----------------------------------------------------------------------------------------
%	SECTION 2
%----------------------------------------------------------------------------------------

\section{Introduction and problem definition}

\section{Algorithm analysis}

\section{Design and implementation of the proposed solution}

\section{Experiments}

\section{Comparison with Fuzzy C-Means}

\section{Organization and development of the coursework}

\section{Conclusions}

\end{document}